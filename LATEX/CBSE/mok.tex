\documentclass[10pt,-letter paper]{article}
\usepackage[left=1in, right=0.75in, top=1in, bottom=0.75in]{geometry}
\usepackage{siunitx}
\usepackage{setspace}
\usepackage{gensymb}
\usepackage{caption}
%\usepackage{subcaption}
\doublespacing
\singlespacing
\usepackage[none]{hyphenat}
\usepackage{amssymb}
\usepackage{relsize}
\usepackage[cmex10]{amsmath}
\usepackage{mathtools}
\usepackage{amsmath}
\usepackage{commath}
\usepackage{amsthm}
\interdisplaylinepenalty=2500
%\savesymbol{iint}
\usepackage{txfonts}
%\restoresymbol{TXF}{iint}
\usepackage{wasysym}
\usepackage{amsthm}
\usepackage{mathrsfs}
\usepackage{txfonts}
\let\vec\mathbf{}
%\usepackage{xtab}
\usepackage{longtable}
\usepackage{multirow}
%\usepackage{algorithm}
\usepackage{amssymb}
%\usepackage{algpseudocode}
\usepackage{enumitem}
\usepackage{mathtools}
%\usepackage{eenrc}
%\usepackage[framemethod=tikz]{mdframed}
\usepackage{listings}
\usepackage[latin1]{inputenc}
\usepackage{textcomp}
\usepackage{titling}
\usepackage{hyperref}
\let\vec\mathbf{}
\usepackage{enumitem}
\usepackage{siunitx}
\usepackage{enumitem}
\usepackage{enumitem}
\usepackage{tfrupee}
\usepackage{amssymb}
\providecommand{\cbrak}[1]{\ensuremath{\left\{#1\right\}}}
\providecommand{\brak}[1]{\ensuremath{\left(#1\right)}}
\providecommand{\sbrak}[1]{\ensuremath{\left[#1\right]}}
\newcommand{\myvec}[1]{\ensuremath{\begin{pmatrix}#1\end{pmatrix}}}
\newcommand{\mydet}[1]{\ensuremath{\begin{vmatrix}#1\end{vmatrix}}}
\providecommand{\mydet}[1]{\ensuremath{\begin{vmatrix}#1\end{vmatrix}}}                         
\providecommand{\myvec}[1]{\ensuremath{\begin{bmatrix}#1\end{bmatrix}}}
\title{MATHEMATICS}
\author{SECTION A}
\date{\today}
\begin{document}
\maketitle
\begin{enumerate}
\section{Vectors}
\item Show that the points $A, B, C$ with position vectors $2 \hat{i}-\hat{j}+\hat{k}$, $\hat{i}-3 \hat{j}-5\hat{k}$ and $3\hat{i}-4 \hat{j}-4 \hat{k}$ respectively, are the vertices of a right-angled triangle. Hence find the area of the triangle.
\item If $\vec{a} = 2\hat{i}-\hat{j}-2\hat{k}$ and $\vec{b} = 7\hat{i}+2\hat{j}-3\hat{k}$, then express $\vec{b}$ in the form of $\vec{b}=\vec{b_1}+\vec{b_2}$, where $\vec{b_1}$ is parallel to $\vec{a}$ and $\vec{b_2}$ is perpendicular to $\vec{a}$.
\section{Geometry}
\item The x-coordinate of a point on the line joining the points $P\brak{2, 2, 1}$ and $Q\brak{5, 1, -2}$ is $4$. Find is $z$-coordinate.
\section{Differentiation}
\item Find the value of $c$ in Rolle's theorem for the function $f\brak{x} = {x}^{3}-3x$ in $\sbrak {-\sqrt{3}, 0}$.
\item If ${x}^{y}+{y}^{x}={a}^{b}$, then find $\dfrac{dy}{dx}$.
\item If ${e}^{y}\brak{x+1=1}$, then show that $\dfrac{{d}^{2} y}{{dx}^{2}}=\dfrac{dy}{{dx}}^2$.
\item find the general solution of the differential equation
\begin{align*}
\frac{dy}{dx}-y=\sin x.
\end{align*}
\item Find the particular solution of the differential equation $\brak{x-y}\frac{dy}{dx} =\brak{x+2y}$, given that $y=0$ when $x=1$.
\section{Integration}
\item Find
\begin{align*} 
\int \dfrac{{\sin}^{2} x-{\cos}^{2} x}{\sin x\cos x}dx
\end{align*}
\item Find:
\begin{align*} 
\int\dfrac{dx}{5-8{x}-{x}^{2}}
\end{align*}
\item Evaluate :	
\begin{align*}
\int_{0}^{\pi} \dfrac{x\tan x}{\sec x+\tan x}dx 
\end{align*}
\item Evaluate :
\begin{align*}
\int_{1}^{4}\brak{\mydet{x-1}+\mydet{x-2}+\mydet{x-4}}dx
\end{align*}
\item Find:
\begin{align*}
\int \frac{{e}^{x}dx}{{{\brak{{{e}^{x}-1}}^{2}}{\brak{{e}^{x}+2}}}}
\end{align*}
\item Using the method of integration, find the area of the triangle $ABC$, coordinates of whose vertices are $A\brak{4,1}, B\brak{6,6}, C\brak{8,4}$
\section{Functions}
\item Determine the value of '$k$' for which the following function is continuous at $x=3$ :
\begin{align*}
f\brak{x}=\left\{\begin{array}{cc}
\frac{(x+3)^2-36}{x-3} & , x \neq 3 \\k & , x=3
\end{array}\right.  
\end{align*}
\item Show the function $f\brak{x} = {x}^{3} -3{x}^{2}+6x-100$ is increasing on $\mathbb
{R}$.
\item Consider $f:\vec{R}-\cbrak{{-\frac{4}{3}}} \rightarrow R-\cbrak{\frac{4}{3}}$ given by $f\brak{x}=\frac{4x+3}{3x+4}$. Show that $f$ is bijective. Find the inverse of $f$ and hence find ${f}^{-1}\brak{0}$ and $x$ such that ${f}^{-1}\brak{x}=2$.
\item Let $A = {Q}\times{Q}$ and let $\ast$ be  a binary operation on $A$ defined by $\brak{a, b}$  $\ast$  $\brak {c, d} = \brak{{a}{c}, b+{a}{d}}$ for $\brak{a, b}, \brak {c, d} \in A$. Determine, whether $\ast$ is commutative and associative. Then, with respect to $\ast$ on $A$.
\begin{enumerate}[label = (\roman*)]
\item Find the identity element in $A$.
\item Find the invertible elements of $A$.
\end{enumerate}
\section{Matrices}
\item If for any ${2}\times{2}$ square matrix $A$, $A\brak{adj A} = \myvec { 8 & 0 \\ 0 & 8 }$, then write the value of $\mydet{A}$.
\item If $A$ is a skew-symmetric matrix of order 3, then prove that det $A$ = 0.
\item Using properties of determinants, prove that
\begin{align*}
\mydet{{a}^{2}+2a & 2a+1 & 1 \\2a+1 & a+2 & 1 \\3 & 3 & 1} = \brak{a-1}^3
\end{align*}
\item Find matrix A such that
\begin{align*}
\myvec{2 & -1 \\1 & 0 \\-3 & 4}A=\myvec{-1 & -8 \\1 & -2 \\9 & 22} 
\end{align*}
\item If $A = \myvec{ 2 & -3 & 5 \\ 3 & 2 & -4 \\ 1 & 1 & -2 }$, then find ${A}^{-1}$ and hence solve the system of linear equations $2x-3y+5z = 11$, $3x+2y-4z = -5$ and $x+y-2z = -3$.
\section{Intersection of conics}
\item Find the area enclosed between the parabola $4y = 3{x}^{2}$ and the straight line $3x-2y+12 = 0$.
\section{Discrete}
\item The volume of a sphere is increasing at the rate of $8{cm}^{3}/s$. Find the rate at which its surface area is increasing when the radius of the sphere is $12 {cm}$.
\section{Probability}
\item A die, whose faces are marked $1, 2, 3$ in red and $4, 5, 6$ in green, is tossed. Let $A$ be the event "number obtained is even" and $b$ be the event "number obtained is red". Find if $A$ and $B$ are independent events.
\item There are $4$ cards numbered $1,3,5$ and $7$, one number on one card. Two cards are drawn at random without replacement. Let $X$ denote the sum of the numbers on the two drawn cards. Find the mean and variance of $X$.
\item Of the students in a school, it is known that $30\%$ have $100\%$ attendance and $70\%$ students are irregular. Previous year results report that $70\%$ of all students who have $100\%$ attendance attain A grade and 10\% irregular students attain A grade in their annual examination. At the end of the year, one student is chosen at random from the school and he was found to have an A grade. What is the probability that the student has $100\%$ attendance? Is regularity required only in school? Justify your answer.
\section{Algebra}
\item Find the distance between the planes $2x-y+2z= 5$ and $5x-2\cdot5y+5z = 2$.
\item If $\tan ^{-1}{\frac{x-3}{x-4}}+{\tan ^{-1}\frac{x+3}{x+4}=\frac{\pi}{4}}$, then find the value of $x$.
\item find the coordinates of the point where the line through the points $\brak{3, -4, 5}$ and $\brak {2, -3, 1}$, crosses the plane determined by the points $\brak{1, 2, 3}$, $\brak{4, 2, -3}$ and $\brak{0, 4, 3}$.
\item A variable plane which remains at a constant distance $3p$ from the origin cuts the coordinate axes at $A, B, C$. Show that the locus of the centroid of triangle  $ABC$ is $\frac{1}{{x}^{2}}+\frac{1}{{y}^{2}}+\frac{1}{{z}^{2}} = \frac{1}{{p}^{2}}$.
\item A window is in the form of a rectangle surmounted by a semicircular opening. The total perimeter of the window is $10m$. Find the dimensions of the window to admit maximum light through the whole opening.
\section{Optimization}
\item Two tailors, $A$ and $B$, earn \rupee $300$ and \rupee $400$ per day respectively. A can stitch $6$ shirts and $4$ pairs of trousers while $B$ can stitch $10$ shirts and $4$ pairs of trousers per day. To find how many days should each of them work and if it is desired to produce at least $60$ shirts and $32$ pairs of trousers at a minimum labour cost, formulate this as an $LPP$.
\item Solve the following linear programming problem graphically :
\\Maximise $Z=7x+10y$
\\subject to the constraints
\begin{align*}
& 4x+6y \leq 240 \\
& 6x+3y \leq 240 \\
& x \geq 10 \\
& x \geq 0, y \geq 0
\end{align*}
\end{enumerate}
\end{document}
